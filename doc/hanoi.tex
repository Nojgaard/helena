The towers of Hanoi is a well-known mathematical game\footnote{The
  description is taken
  from \url{http://en.wikipedia.org/wiki/Tower_of_Hanoi}.}.  It
  consists of three towers, and a number of disks of different sizes
  which can slide onto any tower.  The puzzle starts with the disks
  neatly stacked in order of size on one tower, smallest at the top,
  thus making a conical shape.

The objective of the game is to move the entire stack to another
tower, obeying the following rules:
\begin{itemize}
\item Only one disk may be moved at a time.
\item Each move consists of taking the upper disk from one of the
  towers and sliding it onto another tower, on top of the other disks
  that may already be present on that tower.
\item No disk may be placed on top of a smaller disk.
\end{itemize}

This example illustrates the use of lists in Helena.

\lstinputlisting[frame=single,
caption={Helena file of the towers of Hanoi
(file \texttt{examples/hanoi.lna})},
numbers=left,basicstyle=\small] {../examples/hanoi.lna}

\lstinputlisting[frame=single,
caption={Helena file of the towers of Hanoi properties
(file \texttt{examples/hanoi.prop.lna})},
numbers=left,basicstyle=\small] {../examples/hanoi.prop.lna}
