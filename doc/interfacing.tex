Although the language provided by Helena for arc expressions is quite
rich, it may not be sufficient and the user may, for instance, prefer
to use its own C functions rather than writing these in the Helena
specification.  This is provided by the language through the
\LS{import} construct.  As the code generated by Helena is written in
C all imported components have to be written in this language.  This
chapter first starts with a tutorial illustrating the use of this
feature.
%%%%%%%%%%%%%%%%%%%%%%%%%%%%%%%%%%%%%%%%%%%%%%%%%%%%%%%%%%%%%%%%%%%%%%%%%%%%%%%
\section{Tutorial: Importing C Functions}
Our goal is to simulate the quick-sort algorithm with Helena.  The net
we want to analyze consists of a single transition \LS{swap} that
takes a list of integers from a place \LS{myList}, swaps two of its
elements and put its back in place \LS{myList}.  Each occurence of
this transition simulates a single step of the quick-sort algorithm.
The list we want to sort, \LS{toSort}, is a constant initialized via
function \LS{initList}.  Below is the definition of this net.

\lstinputlisting[frame=single,
caption={Helena file of the sort net
(file \texttt{example/sort.lna})},
numbers=left,basicstyle=\small] {../examples/sort.lna}

The transition \LS{swap} is only firable if the list taken in place
\LS{myList} is not already sorted.  The variable \LS{steps} of
transition \LS{swap} is used to count the number of swaps we have to
perform with quick-sort algorithm.  Note that the function
\LS{quickSort} is always called with list \LS{toSort}.  Hence place
\LS{myList} will successively contain the following token:
\begin{lstlisting}
toSort
quickSort(toSort, 1)
quickSort(toSort, 2)
quickSort(toSort, 3)
...
\end{lstlisting}
until the function returns a sorted list.

Now let us suppose that we do not want the functions used in this net
to be written in Helena but directly in C.  The easiest solution is to
use the \LS{import} feature of Helena.  Note that after these imports,
the bodies of these functions do not have to be declared.  This would
actually be an error.

Lastly, in order to follow the sequence of swaps performed by
quick-sort we write the following line in the property file.
\begin{lstlisting}
state property not_dead:
   reject deadlock;
\end{lstlisting}

In order to write these C functions we need to know how the types of
their parameters have been mapped to C.  This is the purpose of the
\verb+helena-generate-interface+ tool that is invoked with only two
parameters as below:
\begin{verbatim}
helena-generate-interface sort.lna sort_interface.h
\end{verbatim}
File \verb+sort_interface.h+ is the resulting C header file that
contains all declarations that could be required by the user to write
his (her) imported functions.

We only provide here the declarations that are required for the
understanding of this tutorial.  Looking at the net specification, we
need to access the declarations of types \LS{int}, \LS{intList} and
\LS{bool}.  These three types are mapped to the three following types.
\begin{lstlisting}[language=C]
typedef int TYPE_int;

typedef char TYPE_bool;
#define TYPE__ENUM_CONST_bool__false 0
#define TYPE__ENUM_CONST_bool__true 1

typedef struct {
   TYPE_int items[1000];
   unsigned int length;
} TYPE_intList;
\end{lstlisting}
We first notice that each Helena type \LS{myType} is mapped to a C
type \LS{TYPE_myType}.  For the boolean type, \LS{TYPE_bool}, we
notice that its constants \LS{false} and \LS{true} have been mapped to
0 and 1.  The list type \LS{intList} is mapped to a structured type
\LS{TYPE_intList} with two components:
\begin{itemize}
\item\LS{items} is the content of the list stored in an array.  The
  size of this array is equal to the capacity of the list type
  \LS{intList}.
\item\LS{length} is the length of the list, i.e., the number of
  integers in array \LS{items} that are actually part of the list.
\end{itemize}

We are now able to write the three functions that are imported in our
net specification.  Each function has been put in a separate file.
The content of these three files is depicted on
Figure~\ref{fig:imported-code}.  For each imported function
\LS{myFunc} in the net declaration there must be a C function
\LS{IMPORTED_FUNCTION_myFunc}.  The return type and the parameter
types of the C function and the function in the net declaration must
match.  Otherwise a compilation error will occur when invoking Helena.

\begin{figure}
  {\lstinputlisting[language=C,basicstyle=\small,frame=single]{initList.c}}
  {\lstinputlisting[language=C,basicstyle=\small,frame=single]{quickSort.c}}
  {\lstinputlisting[language=C,basicstyle=\small,frame=single]{isSorted.c}}
  \caption{Imported functions of the tutorial}
  \label{fig:imported-code}
\end{figure}

Now that we have written our imported functions we can analyze this
net with Helena.  First, we compile the C code of imported functions
as follows:
\begin{verbatim}
gcc -c initList.c
gcc -c quickSort.c
gcc -c isSorted.c
\end{verbatim}

We can now invoke Helena with option \verb+-L+ in order to specify
which object files must passed to the linker.
\begin{verbatim}
helena -L=initList.o -L=quickSort.o -L=isSorted.o --action=check-not_dead sort.lna
\end{verbatim}

Helena automatically compiles C files generated for the net file
\verb+sort.lna+ and link them with files \verb+initList.o+,
\verb+quickSort.o+ and \verb+isSorted.o+.  After the search is
completed we can see a simulation of the quick-sort algorithm for the
simple list returned by function \LS{initList}:

\begin{lstlisting}
    {
      myList = <( |4, 1, 3, 0, 2|, 1 )>
    }
    (swap, [l = |4, 1, 3, 0, 2|, steps = 1]) ->    
    {
      myList = <( |2, 1, 3, 0, 4|, 2 )>
    }
    (swap, [l = |2, 1, 3, 0, 4|, steps = 2]) ->
    {
      myList = <( |2, 1, 0, 3, 4|, 3 )>
    }
    (swap, [l = |2, 1, 0, 3, 4|, steps = 3]) ->
    {
      myList = <( |0, 1, 2, 3, 4|, 4 )>
    }
\end{lstlisting}
%%%%%%%%%%%%%%%%%%%%%%%%%%%%%%%%%%%%%%%%%%%%%%%%%%%%%%%%%%%%%%%%%%%%%%%%%%%%%%%
\section{The interface file}
As shown in our tutorial, the \verb+helena-generate-interface+ tool
must be invoked in order to generate an header file containing the C
code that could be required to implement imported modules.  The
purpose of this section is to describe exactly the content of this
header file.
\subsection{Generated types}
\begin{table}
  \caption{Mapping Helena types to C}
  \label{tbl:type-interface}
\begin{center}
  \begin{tabular}{|l|l|}
\hline
\multicolumn{1}{|c}{Helena type} &
\multicolumn{1}{|c|}{C type}\\
\hhline{==}
\multicolumn{2}{|c|}{Numeric types}\\
\hline
\begin{lstlisting}
type small: range 0 .. 255;
\end{lstlisting} &
\begin{lstlisting}[language=C]
typedef short TYPE_small;
\end{lstlisting}\\
\hline
\begin{lstlisting}
type big  : range 0 .. 65535;
\end{lstlisting} &
\begin{lstlisting}[language=C]
typedef int TYPE_big;
\end{lstlisting}\\
\hhline{==}
\multicolumn{2}{|c|}{Enumerate types}\\
\hline
\begin{lstlisting}
type color: enum (
   red,
   green,
   blue,
   yellow,
   cyan);
\end{lstlisting} &
\begin{lstlisting}[language=C]
typedef char TYPE_color;
#define TYPE__ENUM_CONST_color__red 0
#define TYPE__ENUM_CONST_color__green 1
#define TYPE__ENUM_CONST_color__blue 2
#define TYPE__ENUM_CONST_color__yellow 3
#define TYPE__ENUM_CONST_color__cyan 4
\end{lstlisting}\\
\hhline{==}
\multicolumn{2}{|c|}{Vector types}\\
\hline
\begin{lstlisting}
type colors: vector[color, bool]
   of int;
\end{lstlisting} &
\begin{lstlisting}[language=C]
typedef struct {
   TYPE_int vector[5][2];
} TYPE_colors;
\end{lstlisting}\\
\hhline{==}
\multicolumn{2}{|c|}{Structured types}\\
\hline
\begin{lstlisting}
type rgbColor: struct {
   small r;
   small g;
   small b;
};
\end{lstlisting} &
\begin{lstlisting}[language=C]
typedef struct {
   TYPE_small r;
   TYPE_small g;
   TYPE_small b;
} TYPE_rgbColor;
\end{lstlisting}\\
\hhline{==}
\multicolumn{2}{|c|}{Container types}\\
\hline
\begin{lstlisting}
type colorList: list[small]
   of color with capacity 5;
\end{lstlisting} &
\begin{lstlisting}[language=C]
typedef struct {
   TYPE_color items[5];
   unsigned int length;
} TYPE_colorList;
\end{lstlisting}\\
\hline
\begin{lstlisting}
type smallSet: set
   of small with capacity 5;
\end{lstlisting} &
\begin{lstlisting}[language=C]
typedef struct {
   TYPE_small items[5];
   unsigned int length;
} TYPE_smallSet;
\end{lstlisting}\\
\hhline{==}
\multicolumn{2}{|c|}{Sub-types}\\
\hline
\begin{lstlisting}
subtype tiny: small
   range 0 .. 15;
\end{lstlisting} &
\begin{lstlisting}[language=C]
typedef TYPE_small TYPE_tiny;
\end{lstlisting}\\
\hline
\begin{lstlisting}
subtype rgColor: color
   range red .. green;
\end{lstlisting} &
\begin{lstlisting}[language=C]
typedef TYPE_color TYPE_rgColor;
#define TYPE__ENUM_CONST_rgColor__red 0
#define TYPE__ENUM_CONST_rgColor__green 1
#define TYPE__ENUM_CONST_rgColor__blue 2
#define TYPE__ENUM_CONST_rgColor__yellow 3
#define TYPE__ENUM_CONST_rgColor__cyan 4
\end{lstlisting}\\
\hline
  \end{tabular}
  \end{center}
\end{table}
The Table~\ref{tbl:type-interface} contains for each kind of type or
sub-type in the Helena net the corresponding C type that is generated.
The translation is pretty straightforward.  We can however make the
following comments:
\begin{itemize}
\item Each Helena type or sub-type \LS{t} is mapped to a C type
  \LS{TYPE_t}.
\item A numeric type is mapped to type \LSC{short} or \LSC{int}
  depending on its range of values.  The same applies for numeric
  types.
\item Each value \LS{val} of an enumerate type \LS{t} is mapped to a
  macro \LSC{TYPE__ENUM_CONST_t__val} that is expanded to the position
  (minus 1) of the value in the list that defines the enumerate type.
  Note that some names are actually quite long.  The reason is that we
  thus avoid name conflicts in the generated code.
\item A vector type \LS{vt} is mapped to a structured type containing
  a single array element called \LSC{vector}.  The dimension(s) of
  this array match(es) with the cardinal(s) of the type(s) used to the
  define the Helena vector type.  In our example, the integer at index
  \LS{[blue,false]} of a C variable \LSC{var} of type
  \LSC{TYPE_colors} can be accessed as follows :
  \LSC{var.vector[TYPE__ENUM_CONST_color__blue][TYPE__ENUM_CONST_bool__false]}.
\item A structured type is mapped to a C \LSC{struct} type that has
  exactly the same structure.
\item A container type \LS{ct} is mapped to a C \LSC{struct} type
  \LSC{TYPE_ct} containing two elements: the items of the list (or
  set) stored in an array \LSC{items}; and, stored in an integer
  component \LSC{length}, the number of items in this array that are
  actually part of the container.  For instance, the Helena expression
  \LS{|red, green, cyan|} of type \LS{colorList} is equivalent to a
  C expression \LS{ex} of type \LSC{TYPE_ct} defined by:
  \begin{itemize}
  \item\LS{ex.items[0] = TYPE__ENUM_CONST_color__red}
  \item\LS{ex.items[1] = TYPE__ENUM_CONST_color__green}
  \item\LS{ex.items[2] = TYPE__ENUM_CONST_color__cyan}
  \item\LS{ex.length = 3}
  \item the value of \LS{ex.items[3]} and \LS{ex.items[4]} are irrelevant.
  \end{itemize}
\item A sub-type is simply translated to its parent type.
\end{itemize}
\subsection{Generated constants and functions}
It may be useful for the user to access in imported modules the values
of some constant(s) or the function(s) declared in the net
specification.  Hence, each constant or function is also accessible in
the header file generated by the \verb+helena-generate-interface+
tool.

An example of translation is provided by
Table~\ref{tbl:const-interface}.  The mapping is straightforward.  We
simply notice that an Helena constant const is mapped to a C variable
\LSC{CONSTANT_const} and that an Helena function func is mapped to a C
function \LSC{FUNCTION_func}.  In addition, the parameter and return
types of the Helena function and the C function must match.
\begin{table}
  \caption{Mapping Helena constants and functions to C}
  \label{tbl:const-interface}
\begin{center}
  \begin{tabular}{|l|l|}
\hline
\multicolumn{1}{|c}{Helena construct} &
\multicolumn{1}{|c|}{C construct}\\
\hhline{==}
\multicolumn{2}{|c|}{Constants}\\
\hline
\begin{lstlisting}
constant rgbColor
   BLUE := {0, 0, 255};
\end{lstlisting} &
\begin{lstlisting}[language=C]
TYPE_rgbColor CONSTANT_BLUE;
\end{lstlisting}\\
\hhline{==}
\multicolumn{2}{|c|}{Functions}\\
\hline
\begin{lstlisting}
function isBlack
          (rgbColor c) -> bool {
   return c.r + c.g + c.b = 0;
}
\end{lstlisting} &
\begin{lstlisting}[language=C]
TYPE_bool FUNCTION_isBlack
(TYPE_rgbColor V2);
\end{lstlisting}\\
\hline
  \end{tabular}
  \end{center}
\end{table}
%%%%%%%%%%%%%%%%%%%%%%%%%%%%%%%%%%%%%%%%%%%%%%%%%%%%%%%%%%%%%%%%%%%%%%%%%%%%%%%
\section{Requirements on imported modules}
Imported functions must fulfill some requirements so that it does not
impact negatively on the behavior of Helena.  These are listed below.
\begin{itemize}
\item An imported function may not have any side effect.  In
  particular, it is absolutely necessary that the function frees all
  memory it allocates.  Otherwise, since the function will be called
  multiple times during the search, memory could be quickly saturated.
\item An imported function must terminate.  This guarantees that the
  search also does.
\item An imported function must be deterministic.  If the function is
  not, Helena is not guaranted to report the same result across
  different executions.  The only exception is for functions that are
  used only once for the initialization of some net constant(s).  The
  value of a constant may for instance be read from a file or from
  user inputs.  Note that all files accessed in imported functions
  must necessarily be accessed via an absolute path.
\end{itemize}
In addition it must also hold that, for each imported function
\LS{func} in the net description, there is in imported module(s) a
function \LSC{IMPORTED_FUNCTION_func} such that its parameter and
return types match with the declaration of function \LS{func} in the
net description.
