\begin{minipage}{\textwidth}
\begin{minipage}{0.1\textwidth}
~
\end{minipage}
\begin{minipage}{0.8\textwidth}
This manual describes Helena, a High LEvel Nets Analyzer.  Helena
verifies properties of high level nets by exploring all these possible
configurations and reports to the user either a success, i.e.  the
property holds, either a faulty execution invalidating the specified
property.  This technique is called model checking, or state space
analysis.  Helena can also perform more basic tasks like state space
exploration in order to report statistics like, e.g., the number of
reachable statistics, the structure of the reachability graph.

Helena is a command line oriented tool freely available under the
terms of the GNU General Public License.  Basic knowledges on Petri
nets, high-level Petri nets and model checking are welcome to
understand this manual.

The installation on a Linux platform is quite simple and should not
raise any problem.  The procedure is detailed in
file \texttt{helena/README}.

This manual is organized as follows.  The specification language of
Helena is presented in Chapter~\ref{chapter_language}.
Chapter~\ref{chapter_using} is devoted to the use of Helena.  Some
examples of the distribution are described in
Chapter~\ref{chapter_examples}.  The possibility of interfacing Helena
with C code is described in Chapter~\ref{chapter_interfacing} together
with a tutorial illustrating this feature.  At last,
Chapter~\ref{chapter_help} is intended to provide some help to the
users and some indications on how to use Helena efficiently.

Appendixes contain the syntax summary and the second version of the
GNU general public license.  An index that references all the
construction of the specification language of the tool can be found at
the end of the document. 
\end{minipage}
\end{minipage}
