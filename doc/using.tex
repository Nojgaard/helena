\newcommand\optionDef[2]{\shortForm{#1}, \longForm{#2}}
%%%%%%%%%%%%%%%%%%%%%%%%%%%%%%%%%%%%%%%%%%%%%%%%%%%%%%%%%%%%%%%%%%%%%%%%%%%%%%%
\section{Invoking Helena}
\label{section_invoking_helena}
Using Helena consists of writing the description of the high level net
in a file, e.g., \texttt{my-net.lna}, and the properties expressed on
this net in a second file, e.g., \texttt{my-net.prop.lna}, and to
invoke Helena on this file.  The command line of Helena has the
following form:
\begin{verbatim}
helena [options] my-net.lna
\end{verbatim}

When invoked, Helena proceeds as follows:
\begin{enumerate}
\item If the net described in file \texttt{my-net.lna} is
  \texttt{my-net}, the directory \texttt{\~{}/.helena/models/lna/my-net}
  is created.
\item A set of C source files and a Makefile are put in directory
  \texttt{\~{}/.helena/models/lna/my-net/src}.
\item These files are compiled and an executable is created which
  corresponds to the actual model checker for the specific net.
\item The compiled executable is launched.
\item Once the search is finished, a report is displayed on the
  standard output.  If a property was checked, this report indicates
  whether the desired property is verified or not.  In the second
  case, a path leading from the initial marking to the faulty marking
  is displayed.
\end{enumerate}

You may find in the \texttt{HELP.md} file of the distribution a
detailed help on all available options.  Alternatively you may invoke
Helena as follows to get this help:
\begin{verbatim}
helena -h=FULL
\end{verbatim}
%%%%%%%%%%%%%%%%%%%%%%%%%%%%%%%%%%%%%%%%%%%%%%%%%%%%%%%%%%%%%%%%%%%%%%%%%%%%%%%
\section{Additional utilities}
\label{section_utilities}
Together with Helena are installed several utilities that we briefly
describe here.
%%%%%%%%%%%%%%%%%%%%%%%%%%%%%%%%%%%%%%%%
\subsection{The \toolName{helena-report} utility}
The purpose of \toolName{helena-report} is to print an XML report that
has been created by Helena.  This utility is useful in the case where
you have already invoked Helena on a net and you do not want to launch
the search again.  Here is an example of use of this utility:
\begin{verbatim}
helena my-net.lna
helena-report my-net
\end{verbatim}
where \texttt{my-net} is the name of the net of file \texttt{my-net.lna}.
The search report will then be printed to the standard output.\\
Alternatively, you can directly pass to \toolName{helena-report} an
xml report previously generated.  For example the following sequence
of commands is equivalent to the previous one:
\begin{verbatim}
helena --report-file=my_report.xml my-net.lna
helena-report my_report.xml
\end{verbatim}
%%%%%%%%%%%%%%%%%%%%%%%%%%%%%%%%%%%%%%%%
\subsection{The \toolName{helena-graph} utility}
\label{sec:helena-graph}
Helena can build the reachability graph of a net in order to display
some statistics on, e.g., its strongly connected components.  This is
the purpose of the \longForm{action=BUILD-GRAPH} option.  This option
is only meaningful if used in conjunction with the
\toolName{helena-graph} utility.  Let us assume that the file
\texttt{my-net.lna} contains the description of net \texttt{my-net}.
A typical use of this combination is
\begin{verbatim}
helena --action=BUILD-GRAPH my-net.lna
helena-graph my-net my_rg_report.pdf
\end{verbatim}
\begin{itemize}
\item The first command explores the reachability graph of the net and
  stores it on disk in the model directory (in \texttt{\~{}/.helena},
  by default).
\item The second command reads this file and produces a report
  containing various informations on the graph e.g., in-/out-degrees
  of nodes, shape of the BFS level graph, SCCs of the graph, dead
  markings, live transitions, \ldots
\end{itemize}
The output format of this report can be pdf or xml.  In the case of a
pdf report, you will need pdflatex as well as the Gnuplot python
library on your system.
%%%%%%%%%%%%%%%%%%%%%%%%%%%%%%%%%%%%%%%%
\subsection{The \toolName{helena-generate-interface} utility}
This tool is used to generate a C header file containing the
translation of types, constants, and functions that can then be used
in imported modules.  Please consult Chapter~\ref{chapter_interfacing}
for further help on this tool.
