We consider in this system a set of $N$ database managers which
communicate to maintain consistent replica of a database.  It is a
well-known and recurrent example of the colored Petri nets literature,
initially presented by Genrich and later by Jensen.

When a manager updates its local copy of the database, he sends
requests to other managers for updating their local copy (transition
{\it Update}). As soon as a manager receives such a request
(transition {\it Receive}) he starts the update of its copy. Its
update finished, each manager acknowledges the initiating manager
(transition {\it Send ack}).  This process finishes when the
initiating manager collects all the acknowledgments (transition {\it
Receive acks}). Managers can be either {\it Inactive}, either {\it
Waiting} for acknowledgments, either {\it Performing} an
update. Places {\it Msgs}, {\it Received}, {\it Acks} and {\it Unused}
model communication channels between sites. Thus, $N.(N-1)$ tokens are
distributed upon these places at each marking. At last the correctness
of the protocol is ensured by place \lstinline{Mutex} which guarantees
that two managers cannot concurrently update their local copy.

\begin{figure}[!h]
\centerline{\scalebox{0.55}{We consider in this system a set of $N$ database managers which
communicate to maintain consistent replica of a database.  It is a
well-known and recurrent example of the colored Petri nets literature,
initially presented by Genrich and later by Jensen.

When a manager updates its local copy of the database, he sends
requests to other managers for updating their local copy (transition
{\it Update}). As soon as a manager receives such a request
(transition {\it Receive}) he starts the update of its copy. Its
update finished, each manager acknowledges the initiating manager
(transition {\it Send ack}).  This process finishes when the
initiating manager collects all the acknowledgments (transition {\it
Receive acks}). Managers can be either {\it Inactive}, either {\it
Waiting} for acknowledgments, either {\it Performing} an
update. Places {\it Msgs}, {\it Received}, {\it Acks} and {\it Unused}
model communication channels between sites. Thus, $N.(N-1)$ tokens are
distributed upon these places at each marking. At last the correctness
of the protocol is ensured by place \lstinline{Mutex} which guarantees
that two managers cannot concurrently update their local copy.

\begin{figure}[!h]
\centerline{\scalebox{0.55}{We consider in this system a set of $N$ database managers which
communicate to maintain consistent replica of a database.  It is a
well-known and recurrent example of the colored Petri nets literature,
initially presented by Genrich and later by Jensen.

When a manager updates its local copy of the database, he sends
requests to other managers for updating their local copy (transition
{\it Update}). As soon as a manager receives such a request
(transition {\it Receive}) he starts the update of its copy. Its
update finished, each manager acknowledges the initiating manager
(transition {\it Send ack}).  This process finishes when the
initiating manager collects all the acknowledgments (transition {\it
Receive acks}). Managers can be either {\it Inactive}, either {\it
Waiting} for acknowledgments, either {\it Performing} an
update. Places {\it Msgs}, {\it Received}, {\it Acks} and {\it Unused}
model communication channels between sites. Thus, $N.(N-1)$ tokens are
distributed upon these places at each marking. At last the correctness
of the protocol is ensured by place \lstinline{Mutex} which guarantees
that two managers cannot concurrently update their local copy.

\begin{figure}[!h]
\centerline{\scalebox{0.55}{We consider in this system a set of $N$ database managers which
communicate to maintain consistent replica of a database.  It is a
well-known and recurrent example of the colored Petri nets literature,
initially presented by Genrich and later by Jensen.

When a manager updates its local copy of the database, he sends
requests to other managers for updating their local copy (transition
{\it Update}). As soon as a manager receives such a request
(transition {\it Receive}) he starts the update of its copy. Its
update finished, each manager acknowledges the initiating manager
(transition {\it Send ack}).  This process finishes when the
initiating manager collects all the acknowledgments (transition {\it
Receive acks}). Managers can be either {\it Inactive}, either {\it
Waiting} for acknowledgments, either {\it Performing} an
update. Places {\it Msgs}, {\it Received}, {\it Acks} and {\it Unused}
model communication channels between sites. Thus, $N.(N-1)$ tokens are
distributed upon these places at each marking. At last the correctness
of the protocol is ensured by place \lstinline{Mutex} which guarantees
that two managers cannot concurrently update their local copy.

\begin{figure}[!h]
\centerline{\scalebox{0.55}{\input{dbm.pdf_t}}}
\caption{The distributed database system}
\label{fig_dbm}
\end{figure}

\lstinputlisting[frame=single,
caption={Helena file of the distributed database system
(file \texttt{examples/dbm.lna})},
numbers=left,basicstyle=\small] {../examples/dbm.lna}

\lstinputlisting[frame=single,
caption={Helena file of the distributed database system properties
(file \texttt{examples/dbm.prop.lna})},
numbers=left,basicstyle=\small] {../examples/dbm.prop.lna}
}}
\caption{The distributed database system}
\label{fig_dbm}
\end{figure}

\lstinputlisting[frame=single,
caption={Helena file of the distributed database system
(file \texttt{examples/dbm.lna})},
numbers=left,basicstyle=\small] {../examples/dbm.lna}

\lstinputlisting[frame=single,
caption={Helena file of the distributed database system properties
(file \texttt{examples/dbm.prop.lna})},
numbers=left,basicstyle=\small] {../examples/dbm.prop.lna}
}}
\caption{The distributed database system}
\label{fig_dbm}
\end{figure}

\lstinputlisting[frame=single,
caption={Helena file of the distributed database system
(file \texttt{examples/dbm.lna})},
numbers=left,basicstyle=\small] {../examples/dbm.lna}

\lstinputlisting[frame=single,
caption={Helena file of the distributed database system properties
(file \texttt{examples/dbm.prop.lna})},
numbers=left,basicstyle=\small] {../examples/dbm.prop.lna}
}}
\caption{The distributed database system}
\label{fig_dbm}
\end{figure}

\lstinputlisting[frame=single,
caption={Helena file of the distributed database system
(file \texttt{examples/dbm.lna})},
numbers=left,basicstyle=\small] {../examples/dbm.lna}

\lstinputlisting[frame=single,
caption={Helena file of the distributed database system properties
(file \texttt{examples/dbm.prop.lna})},
numbers=left,basicstyle=\small] {../examples/dbm.prop.lna}
